\subsection{Project Role}
I served as the Scrum Master for the project and was responsible for creating and programming the scrum board. As the project progressed, the entire team assumed collective responsibility for the board, thereby minimizing the impact of my role.

\subsection{Tasks}
My primary responsibilities encompassed camera programming, user logic (such as environment interaction), and environment logic (which included adding shaders to objects depending on their state). I also managed user input, gathered data from the simulation, and handled data presentation through the UI. Additionally, I was involved in implementing UI design and facilitating communication with the environment. I also programmed a player driven car and additional cameras, though they were not featured in the finished project. The following is a list of my most significant contributions:

\begin{itemize}
    \item \textbf{Camera coding}: I created a state machine for the cameras, programmed the individual cameras, and coded the transition between the cameras when switching from one to another.
    \item \textbf{User logic and input logic}: I coded communication managers between the user and the environment to ensure that the user could interact properly with it. I also created different manager classes for the user that delegated input depending on the state of the simulation.
    \item \textbf{Environment}: I implemented shaders for the roads and cars to highlight user interactions and to indicate the different states of the roads.
    \item \textbf{Graph and UI}: I coded classes that gathered data and communicated this to a graph that I also built. I integrated this graph with the UI and coded and designed various information windows for the UI.
    \item \textbf{User testing}: I designed documents, such as approvals, and conducted user testing.
\end{itemize}

    \ref{Tab:felix-authored-sections}

    \begin{table}[ht]
        \centering
            \begin{tabular}{|c|c|c|}
                %% ----- >>> HEADERS -----
                \hline
                \textbf{Section number} & \textbf{Section name} & \textbf{Co-author}
                \\\hline
                %% ----- <<< HEADERS -----
                %% ----- >>> VALUES -----
                
                1.0 & Introduction (meta text) & 
                \\\hline
                1.2 & Purpose &
                \\\hline
                1.3 & Related work (meta text) & 
                \\\hline
                1.3.1 & Microscopic Traffic Simulations &
                \\\hline
                1.3.2 & ABMU &
                \\\hline
                1.4 & Societal and ethical aspects &
                \\\hline
                2.0 & Theory (meta text) &
                \\\hline
                2.1 & Unity &
                \\\hline
                2.3 & A* Algorithm & Hannes Kaulio
                \\\hline
                2.5 & Scrum and Agile Software Development &
                \\\hline
                2.6 & ABM &
                \\\hline
                2.6.1 & Key features &
                \\\hline
                2.6.2 & Emergence and across-level modeling &
                \\\hline
                2.6.3 & Advantages and applications &
                \\\hline
                2.7 & Design Patterns &
                    \\\hline
                2.7.1 & The Observer Pattern &
                    \\\hline
                2.7.2 & Overview of the Observer Pattern in Unity and C\# &
                    \\\hline
                2.7.3 & Singleton &
                    \\\hline
                2.7.4 & State & 
                    \\\hline
                4.3 & User tests & 
                    \\\hline
                4.3.1 & Generic Tester & 
                    \\\hline
                5.3 & User Testing Feedback &
                    \\\hline
                5.3.1 & Evaluation of the software &
                %% ----- <<< VALUES -----
                \\\hline
            \end{tabular}
        \caption{Felix's authored project report sections}
        \label{Tab:felix-authored-sections}
    \end{table}