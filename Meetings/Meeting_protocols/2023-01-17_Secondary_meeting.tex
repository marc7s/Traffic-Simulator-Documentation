\documentclass{article}
\usepackage{../../preamble}
\usepackage{meeting_protocol_preamble}
\usepackage{fancyhdr}


%%%%% Update this each meeting %%%%%
\newcommand{\meetingdate}{2023-01-17}
\newcommand{\meetingtime}{15:00}
\newcommand{\meetingtitle}{Secondary meeting protocol}
\newcommand{\secretary}{\martin}
\newcommand{\adjuster}{\hannes}
\newcommand{\attendees}{\members}
%%%%%%%%%%%%%%%%%%%%%%%%%%%%%%%%%%%%



  
\begin{document}
    %!!!!! Do not touch !!!!!%
    \begin{titlepage}
        \title{\meetingtitle}
        \date{\meetingdate}
        \maketitle
        
        \thispagestyle{first}
        
        \timeanddate{\meetingdate \space \meetingtime}
        \place{NC415}
        \called{\members}
        \attended{\attendees}
    \end{titlepage}
    \newpage
    \pagestyle{fancy}
    %!!!!!!!!!!!!!!!!!!!!!!!!%

    %%%%% Meeting protocol %%%%%
    \section{Opening}
        

    \section{Notes}
        \begin{itemize}
            \item 
        \end{itemize}

    \section{Discussion}
        During the meeting we discussed the way of working we would choose, and discussed the pros and cons of an agile practice. It was generally regarded as a positive practice if it was kept at a reasonable level, meaning that it should not hinder the work. 

    \section{Decisions}
        We decided to employ an agile approach with two main roles - the scrum master and the project leader. Firstly, a planning meeting should be held where all members align on what the next iteration of the software should be and what features that should include. The time scope for this planning meeting is 2-3 weeks, so the next iteration should represent where the project aims to be in 2-3 weeks. The project leader's task is to, based on the planning meeting, decide during each sprint what the goals and tasks are for this sprint. The project leader is responsible for planning what steps are needed to get to the next iteration, as well as organizing them in the right chronological order. The scrum master on the other hand has the responsibility during each sprint to make sure we succeed with the goals the project leader has set for that sprint. This involves actively communicating with all members and assisting if any part of the development would come to a standstill or encounter a major problem.

        Four other roles were also decided to assign, the first two of which are secretary as well as adjuster for the meetings. The secretary is responsible for taking notes during the meetings as well as setting reminders for the next meeting. The adjuster's task is to approve the protocols and make any needed changes.

        Finally, the last roles are supervisor communicator and project report leader. The supervisor communicator is responsible for the communication with the supervisor, including noting and asking any questions that come up directed at the supervisor, as well as relaying any information the supervisor provides.

        The roles were assigned to the members according to the following list:
        \begin{itemize}
            \item \textbf{Project leader:} \marcus
            \item \textbf{Scrum master:} \felix
            \item \textbf{Secretary:} \martin
            \item \textbf{Adjuster:} \hannes
            \item \textbf{Supervisor communicator:} \jakob
            \item \textbf{Project report leader:} \arvid
        \end{itemize}
    
    %%%%%%%%%%%%%%%%%%%%%%%%%%%%
\end{document}
