\noindent
\label{itm:unity}\textbf{Unity}: Cross-platform game engine.

\noindent
\label{itm:prefab}\textbf{Prefab}: A reusable object in Unity that stores a configuration and can be used as a template for creating assets.

\noindent
\label{itm:agent}\textbf{Agent}: Autonomous systems that inhabits an environment and act based on predefined rules.

\noindent
\label{itm:scrum}\textbf{Scrum}: The scrum agile project management framework provides structure and management of work and is popular among software development teams.

\noindent
\label{itm:pooling}\textbf{Pooling}: A technique used in programming to improve performance by reusing objects instead of creating new ones. 

\noindent
\label{itm:data-structure}\textbf{Data Structure}: A way of organizing and storing data in a computer so that i can be accessed, manipulated, and modified efficiently. Some common examples are arrays, stacks, and linked lists. 

\noindent
\label{itm:abm}\textbf{Agent Based Model (ABM)}: A computer simulation model in which agents interact with each other and their environment to produce emergent behavior. 

\noindent
\label{itm:monobehaviour}\textbf{MonoBehaviour}: Base class for Unity scripts. Provides access to event functions such as Start(), Update(), and so on.  

\noindent
\label{itm:unity-asset}\textbf{Unity Asset}: A file  containing reusable content that can be imported into Unity projects. Can be accessed through Unity's official platform or imported via third-party repositories. 

\noindent
\label{itm:ux}\textbf{UX}: User Experience (UX) refers to the overall experience of the actual user of a product. The goal of good UX design is to create intuitive and enjoyable products.

\noindent
\label{itm:information-visualization}\textbf{Information Visualization}: Field that focuses on creating meaningful and easy to to interpret graphical representations of data.

\noindent
\label{itm:user-testing}\textbf{User Testing}: A method of testing and evaluating a product by observing and gathering data from real users. 







