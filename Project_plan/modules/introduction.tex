% Martin Blom

Traffic congestion is the result of the demand for road/ -and railway travel exceeding the supply. This problem can be seen all around the world\cite{inrix} and it impacts our quality of life. As vehicular traffic builds up, bikes, cars, busses and trams can become stationary or move significantly slower, wasting both time and fuel. Moreover, delaying transportation of goods can lead to huge economical costs, food waste and general inconvenience for effected parties.
\\\\
Not only does this affect our society on the larger scale, but also on an individual scale. The average citizen will spend around a year of their life commuting if the average commuting time doesn't go down. According to Trafikanalys data of traffic habits\cite{trafikanalys_2022}, the average Swedish citizen's daily commuting time during 2019 was just under 1 hour and dropped to around 45 minutes post-COVID. Some amount of commuting time is inevitable in our current society. But if you look at some of the bigger cities in the world such as London, an estimate of 156 hours per person was lost in just traffic delay alone during 2022\cite{inrix}.
\\\\
Other than leading to loss of time and resources, congestion and traffic in general leads to air pollution which poses health hazards and also lowers the quality of life\cite{urban_2004}. A study of air pollution in connection to cars made in the USA during 2022, shows that transportation stood for 27\% of the total greenhouse gas emissions\cite{treehugger_2022}. By reducing the amount of combustion sources that contribute to air pollution we can work on solving multiple problems at once.
\\\\
In section \ref{Purpose} we will present how we aim to create a tool to help with both understanding why, and solving societal, economical and resource problems related to traffic and congestion.