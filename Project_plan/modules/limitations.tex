%% Author: Jakob Windt

To limit the scope of the simulation tool, it was decided to set some boundaries on what the tool should and should not include. 

To begin with, the tool will only simulate vehicles such as cars and buses since including pedestrians was deemed too far fetched because of the added complexity and their small impact on traffic. In addition, there was a worry about the tools overall performance with both vehicles and pedestrians moving around.

Furthermore, there was a discussion about whether the tool would display the map in 2D or 3D. Creating the tool in 2D would take less time to complete, since the overall complexity of the project would decrease. However, by showing the map in only two dimensions; bridges, tunnels, and highway exits/entrances would be harder to interact with compared to in 3D. Also, since the tool is meant to be used by a third party, there is a need to make it appealing. Therefore, the group made the decision to create the tool in 3D, since it allows for more precision when designing road-networks to the user, while also appearing more intuitive.

