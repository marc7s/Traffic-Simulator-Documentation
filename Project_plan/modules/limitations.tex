%% Author: Jakob Windt

To limit the scope of the simulation tool, it is decided that we should set boundaries on what the tool should and should not include. 

To begin with, the tool will only simulate vehicles such as cars and buses since the inclusion of pedestrians would add unnecessary complexity to the simulation. This is because pedestrians have too small an impact on traffic congestion. In addition, having too many movable objects would impact the overall performance of the tool with both vehicles and pedestrians.

Furthermore, a discussion about whether the tool would display the map in 2D or 3D was necessary. Creating the tool in 2D would take less time to complete, since the overall complexity of the project would decrease. However, showing the map in only two dimensions would lead to bridges, tunnels, and highway exits and entrances being harder to interact with when compared with 3D. Also, since the tool is meant to be used by a third party there is a need to make it appealing. On this basis, the group made the decision to create the tool in 3D. It allows for more precision when designing road networks to the user, while also appearing more intuitive.

Lastly, the decision was made to only include a few models of cars with different pollution amounts. These amounts will be similar to Sweden's average emissions by car. It will not be exact because the data will be sourced from the user. The reason behind this is that the average emission depends on the city's exact location, as well as when the data was collected due to its continuous changes.