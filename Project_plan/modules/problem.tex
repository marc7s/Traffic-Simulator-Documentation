In order to achieve a traffic simulation that clearly and visually shows environmental impact as well as traffic congestion, several problems will need to be solved.

%% Felix
There are multiple ways to simulate a traffic flow, however based on the purpose and the established limitations of our project we have chosen to implement an agent based simulation. This would offer a fitting high level of granularity of interplay between the environment and the individual agents. 

%% Hannes (?), Felix
Firstly, the environment to which all \glossaryword{agent}{agents} respond and interact within will have to specified and created. Static environmental objects such as roads and corresponding traffic signs will have to be generated in a modular manner to allow the end user to set up and run their customised simulation.

Furthermore, all individual traffic elements such as cars, traffic lights and buses will need to be simulated. In order to accomplish this, agent-based modeling will be used \cite{agent-based-modeling}. All these traffic elements will be simulated as individual agents  abiding to a set of rules. The challenge will be to decide the different rules and logic for the various agents. Vehicle agents will for example need to interact with the different traffic signs and follow the specified road rules.
%%Felix 
There also exists implicit and more subtle rules that govern movement in public spaces

As a result of using an agent-based approach, performance will become an issue. All agents in the simulation will have to be continuously updated according to their rules. This can become computationally expensive when the number of agents increases. Performance-based design choices will have to be made for city-scale simulations to be possible.

%% Felix
Other issues that will have to be resolved are how to process the order of execution both from a low level perspective such as order of execution of functions within an agent or environment object, as well as on a higher perspective such as order priority amongst different entities in the simulation. Failing to do so might otherwise result in subtle faulty behavior that can be hard to track down the cause of. Unity has some native support for this in the form of different update events with varying priorities. This enables Unity objects, classes inheriting from Unity's base class, monobehaviour, to customize their order of execution. However, it is not clear at the moment whether this will be enough to ensure an easy configuration of execution priority or offer the necessary transparency. Decisions will have to be made if the team should implement our own ABM framework or make use of existing open source unity assets such as ABMU.

%% Felix
Finally, the goal of our project is to offer a user-friendly tool which would require a user interface. This interface needs to clearly communicate which parameters the user can tweak in the simulation and offer an intuitive design. The interface will also need to display relevant information with regards to statistics of the simulation in a manner that is easy to comprehend. Consideration will have to be put into which design patterns to implement to enable this.