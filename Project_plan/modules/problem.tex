In order to achieve a traffic simulation that clearly and visually shows environmental impact as well as traffic congestion, several problems will have to be solved.

Firstly, the environment in which all \glossaryword{agent}{agents} interact will have to be created. Road networks and its corresponding traffic signs have to be generated.

Furthermore, all individual traffic elements such as cars, traffic lights, buses, etc will have to be simulated. In order to accomplish this, agent-based modeling will be used \cite{agent-based-modeling}. All these traffic elements will be simulated as individual agents all abiding to a set of rules. The problem will be to decide all the different rules and logic for the various agents. Vehicle agents will for example need to interact with the different traffic signs and follow the road rules.  

As a result of using an agent-based simulation, performance will surely become an issue. All agents in the simulation will have to be continuously updated according to their rules. This can become computationally expensive when the number of agents increases. Performance-based design choices will have to be made for city-scale simulations to be possible.