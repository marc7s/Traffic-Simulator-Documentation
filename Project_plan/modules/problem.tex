%% Hannes
In order to produce a traffic simulation that clearly and visually shows environmental impact as well as any resulting traffic congestion, several problems will need to be solved.

%% Felix
There are multiple ways to simulate a traffic flow, however based on the purpose and the established limitations of our project we have chosen to implement an \glossaryword{abm}{agent-based model}\cite{agent-based-modeling} (ABM). This would offer a fitting high level of granularity of interplay between the environment and the individual agents. 

%% Hannes, Felix
To start of, the environment to which all \glossaryword{agent}{agents} respond and interact within will have to be specified and created. Static environmental objects such as roads and corresponding traffic signs will have to be generated in a modular manner to allow the end user to set up and run their customised simulation.
%% Hannes
Furthermore, all individual traffic elements such as cars, buses, traffic lights, and roads will need to be simulated. In order to accomplish this, an ABM will be used as mentioned previously. All these traffic elements will be simulated as either individual agents abiding to a set of rules, or environmental objects that relay information to the agents. The challenge will be to decide the different rules and logic for the various agents. Vehicle agents will, for example, need to interact with the different traffic signs and follow the specified road rules.

%%Felix 
There also exists implicit and more subtle rules that govern movement in public spaces, such as the observed phenomenon of personal space radius being dependent upon the velocity of a person in western cultures, achieving its maximum size when the person is stationary\cite{public-spaces}. Similar behavior can be observed in traffic where interactions are not only governed by the laid out traffic rules, but psychological and social factors as well\cite{social-interactions-av}. A challenge therefore lays in sufficiently integrating these variables into the simulation to allow for a realistic depiction of the real life scenario it is trying to emulate. 

%% Hannes
As a result of implementing an ABM, performance will run a high risk of becoming an issue. All agents in the simulation will have to be continuously updated according to their rules. This can become computationally expensive when the number of agents increases. Performance-based design choices will have to be made for city-scale simulations to be possible such as exploring the options for \glossaryword{pooling}{pooling} or constructing customized \glossaryword{data-structure}{data structures}.

%% Felix
Other issues that will have to be resolved are how to process the order of execution both from a low level perspective such as order of execution of functions within an agent or environment object. This includes a higher perspective such as order priority amongst different entities in the simulation. Failing to do so might otherwise result in subtle faulty behavior that can be hard to track down and fix. Unity has some native support for this in the form of different update events with varying priorities. This enables Unity objects, i.e. classes inheriting from Unity's base class \glossaryword{monobehaviour}{MonoBehaviour}, to customize their order of execution. However, it is not clear at the moment whether this will be enough to ensure an easy configuration of execution priority or offer the necessary transparency. Decisions will have to be made if we should implement our own ABM framework or make use of existing \glossaryword{unity-asset}{unity asset} such as the open-source Agent-Based Modelling Framework for Unity3D\cite{abmu}.

%% Felix
The goal of our project is to offer a user-friendly tool which would require a user interface for the end-user to interact with. This interface needs to clearly communicate which parameters the user can tweak in the simulation and offer an intuitive design to convey this. Appropriate \glossaryword{ux}{User Experience} design patterns will have to be reviewed and implemented.

%% Felix
Finally, the interface will also need to display relevant information with regards to statistics of the simulation in a manner that is meaningful and easy to comprehend. To achieve this, we will have to research and make use of concepts from \glossaryword{information-visualization}{Information Visualization} to display relevant information in a comprehensive manner. This will involve using techniques such as data visualization, charting, and graphing to display the statistics of the simulation in a way that is easy to understand and visually appealing.