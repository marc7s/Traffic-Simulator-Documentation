%% Arvid Nyberg + Felix  

Ethical aspects can be broken down into two parts, 1) aspects related to the method of the project, and 2) possible consequences for users of the final product and society as a whole. There are no obvious ethical issues related to the method of our project. For example, there is no use of personal data, no significant security risks, and no use of questionable technology or software. If we choose to introduce any user-testing outside of the development team this will have to be reconsidered and proper implementation of data privacy will be necessary. 

The ethical aspects of the finished product however, is accompanied by more careful considerations. One of the goals of this project is to create a tool that can be used by different end-users of various occupation connected to traffic planning, and offer these users insight about the efficiency and emission associated with different set up of road networks. Since these insights might lead to real-life decisions regarding actual infrastructure, careful consideration will have to be taken regarding the design we choose to implement and what sort of consequences these might have in the finished product. To instill model credibility and prevent model realism bias, we will have to communicate any assumptions and limitation of the model in an easy and accessible manner. ABMs are generally considered challenging with regards to validation and traceability\cite{abm-validation-issues}, and failing to mitigate these might lead to decisions being implemented on obscure premises. 
