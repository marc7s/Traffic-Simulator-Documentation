% WRITTEN BY JAKOB WINDT, 2023
This section will summarize the project. It goes into detail on how future problem presentations might arise, and how the tool might be configured to be used in other fields. Lastly, it will quickly summarize some of the skills that were learned during the project.

\section{Project Conclusion}
    % WRITTEN BY MARTIN BLOM, 2023
    \subsection{Additional Knowledge}
        There are two main areas where additional knowledge would have the most impact on the final product. The first area is revolving around programming and the game engine that was used. To start of a big majority of the group members had never worked on a project of this size nor in a team this big. Without a lot of experience we quickly found it hard to keep progress organized and the repo became cluttered. This did in some cases force us to backtrack and loose progress in same time. With more experience and prior knowledge in working with larger software projects, the workflow would have been smoother and mistakes could have been avoided.

        The other area where more knowledge would have been beneficial is surrounding infrastructure engineering, traffic flow modeling and ABM systems. Having more knowledge on these subjects would have allowed for increased accuracy and faster development of the vehicle systems as well as helped with how to implement their rule sets. Moreover, it would have given us better knowledge on what statistic variables are important and the best way to present them to the user.

    % WRITTEN BY MARCUS SCHAGERBERG, 2023
    \subsection{Future Problem Presentations}
        Given the insights gained throughout the development process, another area that might benefit from a tool like this would be for airports. They, like the road networks, have traffic they need to handle and require planning to construct airports with large capacities. They have taxiways where the airplanes move between the gates and the landing strip. These relocating planes interfere with other planes trying to take off or land, which means a good traffic flow is a requirement for a functioning large capacity airport. A simulation tool like this one could be tailored towards airports, investigating the different synergies that affect the performance of an airport. The tool should also allow for different airport configurations to be simulated in order to evaluate them. Airport traffic brings with it different challenges and relationships. One example is that all airport traffic is scheduled, which means that well planned arrival and departure times can improve the performance and increase the capacity and throughput of an airport.

        As the future looks to bring more and more autonomous vehicles, transportation planning tools could play an even more important role. If all vehicles are autonomous, the traffic could be centrally controlled to avoid congestions and balance the load in the road system to optimise its performance. A tool like this could be extended or repurposed in order to analyse these future traffic flows. New possibilities also emerge, since solutions could be based around the fact that the traffic is autonomous and can therefore easily be redirected better utilise the available road network. Perhaps it would be beneficial to redirect some vehicles on slight detours, making the travel time slightly longer. This could ease the traffic in some areas, and better distribute the traffic to overall decrease the travel times in the road system as a whole.


    % WRITTEN BY JAKOB WINDT, 2023
    \subsection{Future Usage of Simulation}
        In the future, many new use cases can be developed for the current simulation. An example for this could be to convert the simulation to something video game like allowing users to select a real-life location, and then drive around in that location with realistic traffic. Other than the previously mentioned improvements that needs to be done to OSM generation, and intersection types, the only thing that would have to be implemented is a way for a user to control and drive a vehicle on the road. This could quickly be implemented since the simulation is created in Unity, with its vast support for user inputs and camera control. 

        Similarly, the simulation could be converted to a tool that would help people learn how to drive. In the same way, first person controls and cameras could be added, which could allow the user to use driving simulator hardware such as a steering wheel, pedals, and a gear shifter. This could be beneficial to driving schools since it could assist the students during the early stages of learning how to drive a car. Since the student would learn the basics in the simulator, the risk that comes when driving a real car from the beginning could be removed. 

    
    % WRITTEN BY JAKOB WINDT, 2023
    \subsection{Learning Outcome}
        If we would redo the project from scratch, we would've changed a few aspects of our approach. To begin, we would have lowered the scope of our project so we instead could focus on improving the quality. This would result with a simulation with fewer usage cases, but with better core functionality. This would allow the project to grow quicker with more features in the long term, due to the added stability. Another aspect that would technically change, is that our custom Road Generator would exist. With access to this asset from day one, we could instead spend our time focusing on adding new features, and improving the simulation at its whole. 

        In the end, this project resulted in us learning numerous new skills. In the beginning of the project, only a few of us had used Unity before, and only to a limited state, and now we're all knowledgeable within the game-development platform. In a similar way, this is reflected in many parts of the project, including and not limited to, Git, the scrum framework, C\#, road-network design, and most importantly working efficiently as a software development team. 
