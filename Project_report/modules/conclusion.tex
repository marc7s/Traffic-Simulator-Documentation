% WRITTEN BY JAKOB WINDT, 2023
This section will summarize the project. It goes into detail on how future problem presentations might arise, and how the tool might be configured to be used in other fields. Lastly, it will quickly summarize some of the skills that were learned during the project.

    % WRITTEN BY MARCUS SCHAGERBERG, 2023
    \section{Distinction from other tools}
        There are a few key points differentiating this tool from most others. Firstly, it is the focus on accessibility, allowing a much wider audience to use the tool compared to most existing solutions that are focused around civil engineer usage. One advantage of this is that it enables regular citizens to effectively give their own input and suggestions for improvements in their city.

        Since the purpose of the project was to develop an accessible traffic simulation tool, the input from the user testing sessions was vital in order to confirm that this was achieved. Through the combination of a tester with prior experience and training in an existing tool and the generic testers that gave valuable feedback regarding the user-friendliness, it was clear that the tool was accessible and easy to use. Constructive feedback was also provided, which was used to improve the tool. An important part of this, as well as a distinction from most other tools, is the 3D environment instead of the more widely used 2D alternative. This makes it harder to understand large road systems, but was identified to be an improvement for smaller networks as well as making the tool more accessible according to the testers.

    % WRITTEN BY MARTIN BLOM, 2023
    \section{Additional Knowledge}
        There are two main areas where additional knowledge would have had the most impact on the final product. The first area revolves around programming, and the game engine that was used. To start off, a large portion of the group members had never worked on a project of this size, nor in a team this big. With a lack of experience, we quickly found it hard to keep progress organized which resulted in the repository becoming cluttered. With more experience and prior knowledge in working with larger software projects, the workflow would have been smoother, and mistakes could have been avoided.

        The other area where more knowledge would have been beneficial, is surrounding infrastructure engineering, traffic flow modeling and ABM systems. With some background in these subjects, it would allow for increased accuracy and quicker development of the vehicle simulation system. Moreover, it would have given us a better understanding on which statistics are most important, and the best way to present them to the user.

    % WRITTEN BY MARCUS SCHAGERBERG, 2023
    \section{Future Problem Presentations}
        Given the insights gained throughout the development process, another area that might benefit from a tool like this would be airports. They, like the road networks, have traffic they need to handle, and require planning to construct airports with large capacities. They have taxiways where the airplanes move between the gates and the landing strip. These relocating planes interfere with other planes trying to take off or land, which means that a good traffic flow is a requirement for a well-functioning large scale airport. A simulation tool like this one could be tailored towards airports, investigating the different synergies that affect the performance of an airport. The tool should also allow for different airport configurations to be simulated in order to evaluate them. Airport traffic brings with it a different set of challenges and relationships. One example is that all airport traffic is scheduled, which means that well planned arrival and departure times can improve the performance, increase the capacity and the overall throughput of an airport.

        As the future looks to bring more and more autonomous vehicles, transportation planning tools could play an even more important role. If all vehicles are autonomous, the traffic could be centrally controlled to avoid congestions, and balance the load in the road system to optimise its performance. A tool like this could be extended or repurposed in order to analyse these future traffic flows. New possibilities also emerge, since solutions could be based around the fact that the traffic is autonomous, and can therefore easily be redirected to better utilise the available road network. Perhaps it would be beneficial to redirect some vehicles on slight detours, making the travel time slightly longer. This could ease the traffic in some areas, and better distribute the traffic, to overall decrease the travel times in the road system as a whole.


    % WRITTEN BY JAKOB WINDT, 2023
    \section{Future Usage of the Simulation}
        In the future, many new use cases can be developed for the current simulation. An example for this could be to convert the simulation to a video game, allowing users to select a real-life location, and then drive around in that location with realistic traffic. Other than the previously mentioned improvements, the only feature that would have to be implemented is a way for a user to drive a vehicle on the road. This could quickly be implemented due to Unity's vast support for user inputs and camera control.

        Similarly, the simulation could be converted to a tool that would help people learn how to drive. In the same way, first person controls and cameras could be added, which could allow the user to use driving simulator hardware such as a steering wheel, pedals, and a gear shifter. This could be beneficial to driving schools since it could assist the students during the early stages of learning how to drive a car. The student would learn the basics in the simulator, without the risk or cost that comes with driving a real car in traffic with vehicles.
    
    % WRITTEN BY JAKOB WINDT, 2023
    \section{Learning Outcome}
        If we were to redo the project from scratch, we would have changed a few aspects of our approach. To begin, we would have lowered the scope of the project, so that we instead could focus on improving the quality. This would result in a simulation with fewer use cases, but with better core functionality. Therefore allowing the project to grow exponentially with more features in the long term, due to the added stability. Another aspect that would change, is that our custom road generator tool would be prioritized to be implemented. With access to this asset from the beginning, we could instead spend our time focusing on adding new features, and improving the simulation at its core.

        In the end, this project resulted in us learning numerous new skills which would further our expertise as we continue to grow and learn. In the beginning of the project, only a few of us had used Unity before, and only to a limited state. Now, we are all knowledgeable within the game development platform. In a similar way, this is reflected in many parts of the project, including but not limited to; Git, the scrum framework, C\#, road network design, and more importantly, working efficiently as a software development team.
