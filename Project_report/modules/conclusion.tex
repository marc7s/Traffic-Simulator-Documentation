\section{}
    \subsection{Additional Knowledge}
        

    \subsection{Future Problems}
    % KRAV ENLIGT CANVAS


    % WRITTEN BY JAKOB WINDT, 2023
    \subsection{Future Usage of Simulation}
        In the future, many new use cases can be developed for the current simulation. An example for this could be to convert the simulation to something video game like allowing users to select a real-life location, and then drive around in that location with realistic traffic. Other than the previously mentioned improvements that needs to be done to OSM generation, and intersection types, the only thing that would have to be implemented is a way for a user to control and drive a vehicle on the road. This could quickly be implemented since the simulation is created in Unity, with its vast support for user inputs and camera control. 

        Similarly, the simulation could be converted to a tool that would help people learn how to drive. In the same way, first person controls and cameras could be added, which could allow the user to use driving simulator hardware such as a steering wheel, pedals, and a gear shifter. This could be beneficial to driving schools since it could assist the students during the early stages of learning how to drive a car. Since the student would learn the basics in the simulator, the risk of driving a real car from the beginning could be removed. 

    
    % WRITTEN BY JAKOB WINDT, 2023
    \subsection{Learning Outcome}
        % Throughout the project, a few things could have been improved..

        If we would redo the project from scratch, we would've changed a few aspects of our approach. To begin, we would have lowered the scope of our project so we instead could focus on improving the quality. This would result with a simulation with fewer usage cases, but with better core functionality. This would allow the project to grow quicker with more features in the long term, due to the added stability. Another aspect that would technically change, is that our custom Road Generator would exist. With access to this asset from day one, we could instead spend our time focusing on adding new features, and improving the simulation at its whole. 

        In the end, this project resulted in us learning numerous new skills. In the beginning of the project, only a few of us had used Unity before, and only to a limited state, and now we're all knowledgeable within the game-development platform. In a similar way, this is reflected in many parts of the project, including and not limited to, Git, the scrum framework, C\#, road-network design, and most importantly working efficiently as a software development team. 
