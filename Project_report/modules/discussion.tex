% UPDATED BY MARCUS SCHAGERBERG, 2023
% CREATED BY WOLFGANG AHRENDT, 2021

% WRITTEN BY MARCUS SCHAGERBERG, 2023
\section{Unreached Goals}
    Some features took more resources and longer time to implement than initially estimated, which limited development in other areas. The most significant example of this is the road creation system. When defining the goals and planning the project, an asset called RoadGenerator had been found that supported this \cite{road-architect}. The asset was well-known and had an extensive documentation. However, when starting the development using this asset, several severe bugs were identified. We think the issue is that it was developed for an earlier release of Unity, and was not compatible with our version even though there were no documentation of this. RoadArchitect supported creation of roads and intersections with many features, such as traffic lights, road markings, railings and bridges. This was a major setback as we had to develop our own road creation tool. Adding all the needed features to the road creation tool was an extensive process that would have been avoided if we could use RoadArchitect as it had nearly all the features needed for this project.

    An area which was less prioritised due to the time constraints was the support for public transport, which was only added with basic support. The goal was to have a working public transportation system with the ability to change a parameter in the simulation for how much of the traffic is handled through it. This would allow users to find the optimal balance of public transport to cars.

    A map editor was supposed to be developed to allow users to create their own maps and road systems to analyse. The time and change in priorities did not allow for the time to add this. This is definitely something that would need to be added in the future if the tool would be released. As of now it is only possible to edit the maps in the built-in Unity editor, which is not available when generating an executable version of the tool.

    % % WRITTEN BY MARTIN BLOM, 2023
\section{Future Improvements}
    This section will present any future improvements or changes that we would like to implement. We will partly provide motivations as to why these improvements were not made and if unclear, why we deem them important. Improvements without any explanation as to why they are missing can be presumed to be a result of lack of time.
    
    % WRITTEN BY JAKOB WINDT, 2023
    \subsection{Road and Intersection Types}
        For intersections, the simulation only supports three and four way intersections. These intersections can be created with either one-way or regular two lane roads, or a mix of both. In reality, there are other intersection types such as the roundabout that there currently is no support for. For this to be implemented in the future, the yielding of the cars has to improve from their current state. They current lack the ability to yield within roundabout type intersections.
    
        Furthermore, when designing large road networks, especially outside cities, highways are critical. Highways are currently not supported in the simulation because of a few different reason. To begin, there is functionality for multiple lanes when creating roads within the road generator. However, the vehicles are not able to switch lanes in the simulation, making multiple lanes unusable. Since lane switching isn't supported, vehicles don't have the ability to overtake each other which occurs on highways. There are also no highway entrances and exits in the simulation, which would need to be implemented for highway roads to function correctly.

    % WRITTEN BY JAKOB WINDT, 2023
    \subsection{OSM}
        In the current state of the simulation, OSM imports are supported, but not in the best stage. Currently, OSM datasets can be imported in the pre-build of the simulation. This means that to generate a real-life location into the simulation, the user would have to manually import the OSM file into the Unity project, and then change the current path to the file within the code. In the future, there should be functionality to either import OSM data directly from within the simulation, or a way to import OSM files without having to manually edit code.
        
        Additionally, OSM imports are in no way perfect at its current stage. Because of the lack of support for intersection types, the road network can end up generating in an unfinished state. In the same way there have been other issues, for example, when two intersections are located next to each other, one of them usually fails to generate due to overlap between the intersection nodes. Next, not all data from the OSM file is used when generating the world. In addition to the current POIs outlined in \ref{poi}, one data point that should be implemented in the future, are gas and charging stations.

    % WRITTEN BY JAKOB WINDT, 2023
    \subsection{Vehicle Types}
        With the previously mentioned charging and gas stations, different vehicle types representing electric and gas driven could be included. This would improve the statistical accuracy of the simulation, allowing the user to modify the differential between the two types. The reason this should be implemented is because of the increase in electric vehicles registered in the world, and their effect on emissions \cite{IEA}. 
        
        Equally important, other forms of transport should be included as well. These would include trains, trams, and taxi's which are commonly found within, or nearby cities. With the inclusion of these vehicles, the amount of cars in a road network should decrease due to the use of public transport. However, this depends on the efficiency, availability, and cost of said public transport. With the reduced amount of cars in the road network, both efficiency will increase, and total emissions should decrease.

    % WRITTEN BY MARCUS SCHAGERBERG, 2023
    \subsection{Performance Optimisation}
        One of the major limitations of the tool is the performance. Simulating individual agents is performance intensive. The physics engine used by the vehicles to provide realistic handling uses a lot of resources, especially when simulating many vehicles. The performance mode helps with this, but would need to be developed further to allow larger networks to be simulated with reasonable performance. The mathematical calculations it currently uses could be optimized or perhaps even simplified while still fulfilling the requirements.

        A great way of achieving a significant increase in performance is by multithreading heavy workloads. Nowadays almost all computer processors have multiple threads allowing them to perform tasks in parallell. Some workloads are more difficult to parallellize as multiple cores simultaneously reading and writing from the same memory locations can interfere with each other. A heavy task that could easily benefit from multithreading is the OSM import, where there is a clear separation between several import stages. This would allow the maps to be generated quicker, especially on less powerful hardware. It is also possible to multithread the vehicle physics calculations, however this is much harder as they interact with each other. Therefore, all vehicles need to be updated before the program can move on, which will limit the improvement of this optimisation.


    % WRITTEN BY MARCUS SCHAGERBERG, 2023
    \subsection{Simulation Improvements}
        One of the areas where there is a lot of potential for improvement is related to the simulation itself. The simulation could be made more realistic by improving the target assignment algorithm, which determines where the vehicles should navigate to. The algorithm can be expanded to include parameters such as the time of day to determine what the vehicles should do, which would allow the implementations of rules causing many vehicles to drive from houses to companies and industrial areas in the morning and returning in the evening.
        
        In addition to the buses following routes and stopping at the bus stops, their behaviour could be extended to follow schedules with a limited pool of available buses. This would limit the amount of people that could use the public transport system, causing more vehicles to appear in the network. Expanding on the public transport integration is something that could make the tool even more useful and an innovative area, as noted during the user tests where other tools lack this feature. This could be used for finding the optimal balance between public transport and personal vehicles and aid in transportation planning.

        Improving the vehicle driving would allow more road systems to be created, as not all types of roads and rules are implemented. The tool only supports three way and four way intersections currently, lacking support for roundabouts and intersections with more than four roads. Expanding the support for different types of road signs and traffic rules would also be an improvement over other tools that in most cases does not support this. An example of this is how our tool support the priority to the right rule, which could mean that in certain situations an intersection could be easily congested due to vehicles on one of the roads always having to yield to those on the other. Additional signs such as those defining rules for parkings could also affect the simulation, making certain parking spaces restricted for example during certain times of day or for non-residents. Due to the nature of the tool being low level with simulation of individual vehicles, there is room for improvement with features like these which would separate the tool even more on the market, creating a unique feature set.

    % WRITTEN BY MARTIN BLOM, 2023
    \subsection{Statistics}
        Statistics is one of the areas where improvements can almost always be made seeing as the understanding of the presented data is subjective. Therefore, many new ways of visualizing the same data can put fourth to widen the audience that can make use of it. Moreover, with the large amounts of data that can be produced many different comparisons can be made. Currently implemented is statistics and graphs that we deemed the most important, such as fuel-consumption over time or congestion over time. Al tough, any variable such as the fuel-consumption could also be viewed in different domains such as over traveled distance or speed. With the multitude of different ways of representing data against each other together with including new variables, the possibility of adding more statistics is almost endless.

        Statistics could also be improved purely visually by adding filters or features that highlight or better indicate the data that the user is searching for. An example would be adding more visual highlighters; in the same fashion that the road is highlighted red during high congestion and green during low. For other variables such as emissions or total usage/wear. 

    % WRITTEN BY MARTIN BLOM, 2023
    \subsection{Map Editor}
        An area where the tool is currently lacking is the building of the network. As it stands the only way to manipulate the environment, POIs, roads and other aspects of infrastructure is through the built-in Unity editor. This puts a barrier for users without any knowledge of Unity and might discourage them from further using the tool. Even with understanding on how to do this, it is very tedious and time consuming process. To solve this a built in map editor would be developed, allowing the user to move, manipulate, create and delete any of the mentioned aspects inside the tool. This would dramatically improve both the time spent and the simplicity of the process, further boosting the tools efficiency and user-friendliness.

\section{User Testing Feedback}
    % WRITTEN BY MARCUS SCHAGERBERG, 2023
    \subsection{Evaluation of the software}
        According to the user feedback, we found our tool to be intuitive and easy to use.

    % WRITTEN BY MARCUS SCHAGERBERG, 2023
    \subsection{Changes made based on the testing feedback}
        In order to eliminate the ambiguity between the two statistics related buttons, it was decided to remove one of them and instead dynamically change the content based on the context. It was changed so that if the statistics panel is open while a vehicle is selected, it displays the individual statistics related to that vehicle. Otherwise, it displays the overall statistics for the entire road network. This solution was proposed to the tester and the response was that it would be a good solution.

        As some users felt that it was a bit cumbersome to navigate around, the camera controls was changed so that the keyboard is used for movement around the plane, and the mouse to rotate and pitch the camera. This change made the tool in line with most games as well as other software such as CAD programs, improving the accessibility and intuitiveness as users can recognise the controls from previous experiences. The movement speed was also changed to depend on the zoom level, so that the camera moves slower when zoomed in on vehicles or intersections while being faster when positioned higher above the ground. This allows the user to quickly move between areas of the network while still being able to have fine control over the movement when zoomed in.

        A button was also added to toggle the colour coding feature that can colour the roads based on current statistics, such as congestion levels or emissions. This is helpful for visualising the current performance of the road network and to quickly identify problematic areas when viewing the system from a macro perspective.
    
    
% WRITTEN BY JAKOB WINDT, 2023 --TYP KLAR
\section{Development Process}
    As previously mentioned in \ref{sub:weekly-sprints}, the decision was made to follow the scrum framework. This allowed the group to set clear goals for each week, which simplified the process of dividing work. For the most part, this worked as well as expected. However, something that was noticed during the project was that the backlog was slowly growing in amount of tickets. Each week, a small number of tickets were delayed for different reasons, which is the cause of the problem. 

    During the development of the simulation, code reviewing was an important factor in making sure the code was correct, commented, and well written. This worked well for the majority of the time, but minor problems arose for larger PR's. The reason behind this is due to the problem that when someone tries to merge a large amount of code at the same time, the time and effort required by the code reviewer increases. This is because the reviewer first has to try and understand the logic and functionality of the code itself. Therefore, if a code reviewer is tasked with reviewing a large PR, the reviewer might not have the same attention to detail compared to when reviewing a lower amount of code, resulting in a less effective review. 



        