% UPDATED BY MARCUS SCHAGERBERG, 2023
% CREATED BY DAVID FRISK, 2016

%\oneLineTitle\\
%\oneLineSubtitle\\
%\MEMBERTILDELIST\\

%Department of Computer Science and Engineering\\
%Chalmers University of Technology and University of Gothenburg\setlength{\parskip}{0.5cm}

% Supress header
\thispagestyle{plain} 
\setlength{\parskip}{0pt plus 1.0pt}
% WRITTEN BY MARTIN BLOM, 2023
\section*{Abstract}
    \subsection*{Abstract 1}
        Tools for simulating traffic are important for transportation planning and to be able to analyse how road network changes will affect the traffic flows. This project aims to develop such a tool, with the distinction from existing tools that it should be intuitive to use and have a high accessibility for users. This thesis describes the development process and the methods used when creating the tool. It also dives into the different features it has and how they contribute to creating a realistic traffic simulator. From the user tests of the tool, it was found that it was intuitive and easy to use, but lacked in feature density and simulation tools when compared to existing tools.

    
    \subsection*{Abstract 2}
        With the rise of large and densely populated cities, increased vehicular traffic is becoming more usual and with it comes an expected increase in traffic delays. We are all affected by this in one way or another, be it by personal transportation, resource shipping or any other situation. Moreover, traffic delays cost companies and governments large amounts of money and resources as well as increased emissions with reshipping or engines idling. The aim of this bachelor thesis is to combat these problems by developing a traffic simulation tool, used to analyse road network performance in the hopes to foresee probable issues and help with finding solutions.
    
        The simulation tool will be developed within Unity using the programming language C\# and will allow users to input traffic parameters such as traffic volume, vehicle types, and road infrastructure details. One of the main goals is for the tool to be intuitive and easy to operate for users. Therefore, a lot of weight is put into developing a user-friendly presentation of statistics. To help with achieving this, user-testing will be carried out and any interfaces in need of change will be tweaked and refined according to the feedback. Moreover, the tool will offer real-time visualization of the simulation, which is not often found in similar tools. Finally the results will be visualized using graphs and charts to provide a comprehensive analysis of traffic flow, including aspects such as traffic density, average speed, and queue length.
    
        The finished product will not be a fully realistic simulator, but a robust base software with good potential to be built upon and improved. To promote this, the whole project will be made open source making it accessible to anyone wanting to use or further develop it.

\newpage
% WRITTEN BY MARTIN BLOM, 2023
\section*{Sammandrag}
    Med en ökning av städer med hög befolkningstäthet blir ökad fordonstrafik vanligare och med det följer förväntade ökningar av trafikstörningar. Vi påverkas alla av detta på ett eller annat sätt, antingen genom persontransport, resursleveranser eller i andra situationer. Dessutom kostar trafikstörningar företag och regeringar stora mängder pengar och resurser, samt leder till ökade utsläpp genom omleveranser eller tomgångskörning av motorer. Syftet med denna kandidatuppsats är att bekämpa dessa problem genom att utveckla ett trafiksimuleringsverktyg som används för att analysera prestandan i vägnätet i förhoppningen om att förutse potentiella problem och hjälpa till att hitta lösningar.

    Simuleringsverktyget kommer att utvecklas inom Unity med programmeringsspråket C\# och tillåta användare att mata in trafikparametrar så som trafikvolym, fordonstyper och väginfrastrukturinformation. Ett av huvudmålen är att verktyget ska vara intuitivt och lätt att använda. Därför läggs stor vikt vid att utveckla en användarvänlig presentation av statistik. För att hjälpa till med detta kommer användartester att utföras och alla gränssnitt som behöver ändras kommer att justeras och förfinas enligt feedbacken. Dessutom kommer verktyget att erbjuda realtidsvisualisering av simuleringen, vilket ofta inte finns i liknande verktyg. Slutligen kommer resultaten att visualiseras med grafer och diagram för att ge en omfattande analys av trafikflödet, inklusive aspekter så som trafiktäthet, genomsnittlig hastighet och kölängd. 

    Den färdiga produkten kommer inte att vara en fullständigt realistisk simulator, utan en robust grundprogramvara med god potential att förbättras och byggas vidare på. För att främja detta kommer hela projektet att göras till öppen källkod så att det är tillgängligt för alla som vill använda eller vidareutveckla det.

% KEYWORDS (MAXIMUM 10 WORDS)
\vfill
Keywords: traffic, simulation, flow, emissions, vehicles, driving

% Create empty back of side
\newpage
\thispagestyle{empty}
\mbox{}