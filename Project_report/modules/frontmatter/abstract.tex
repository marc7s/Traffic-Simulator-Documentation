% UPDATED BY MARCUS SCHAGERBERG, 2023
% CREATED BY DAVID FRISK, 2016

%\oneLineTitle\\
%\oneLineSubtitle\\
%\MEMBERTILDELIST\\

%Department of Computer Science and Engineering\\
%Chalmers University of Technology and University of Gothenburg\setlength{\parskip}{0.5cm}

% Supress header
\thispagestyle{plain} 
\setlength{\parskip}{0pt plus 1.0pt}
% WRITTEN BY MARTIN BLOM AND MARCUS SCHAGERBERG, 2023
\section*{Abstract}
    With the rise of large and densely populated cities, increased vehicular traffic is becoming more common and with it comes an expected increase in traffic delays. The aim of this bachelor's thesis is to combat these problems by developing a traffic simulation tool, used to analyse road network performance in the hopes to foresee probable issues and help with finding solutions. The simulation tool was developed within Unity, using the programming language C\# and allowed users to input traffic parameters such as traffic volume, vehicle types, and road infrastructure details. One of the main goals was for the tool to be intuitive and easy to operate for users. Therefore, a lot of weight was put into developing a user-friendly presentation of statistics. To help with achieving this, user tests were carried out and any interfaces in need of change were tweaked and refined according to the feedback. Moreover, the tool offered real-time visualisation of the simulation, which were often not found in similar tools. Finally, the results were visualised using graphs and charts to provide a comprehensive analysis of traffic flow, including aspects such as traffic density, average speed, and fuel consumption.
    \\\\
    The results from the user tests showed that the tool was intuitive and easy to use, however lacked in feature density and simulation abilities when compared to already existing tools. To compete with these tools, the simulator would need a simplified mode for simulating large road networks.

\newpage
% WRITTEN BY MARTIN BLOM AND MARCUS SCHAGERBERG, 2023
\section*{Sammandrag}
    Med en ökning av städer med hög befolkningstäthet blir ökad fordonstrafik vanligare, och med det följer förväntade ökningar av transporttider. Syftet med denna kandidatuppsats är att bekämpa dessa problem, genom att utveckla ett trafiksimuleringsverktyg som kan användas för att analysera prestandan i ett vägnät. Detta i förhoppningen om att förutse potentiella problem och bidra med att hitta lösningar. Simuleringsverktyget utvecklades i Unity med programspråket C\# och tillät användare att mata in parametrar som trafikvolym, fordonstyper och väginfrastrukturinformation. Ett av huvudmålen var att verktyget skulle vara intuitivt och lätt att använda. Därför lades stor vikt vid att utveckla en användarvänlig presentation av statistiken. För att hjälpa till med detta utfördes användartester och alla gränssnitt som behövde justeras förbättrades enligt feedbacken. Dessutom erbjöd verktyget realtidsvisualisering av simuleringen, vilket ofta saknades i liknande verktyg. Slutligen visualiserades resultaten med grafer och diagram för att ge en omfattande analys av trafikflödet, inklusive aspekter som trafiktäthet, genomsnittlig hastighet och bränsleförbrukning. 
    \\\\
    Resultaten från användartesterna visade att verktyget var intuitivt och lätt att använda, dock saknades en del funktionalitet i jämförelse med redan tillgänglig programvara. För att konkurrera med dessa verktyg skulle simulatorn behöva en förenklad simulering för att hantera stora vägnätverk.


% KEYWORDS (MAXIMUM 10 WORDS)
\vfill
Keywords: traffic, simulation, flow, emissions, vehicles, driving, transportation planning

% Create empty back of side
\newpage
\thispagestyle{empty}
\mbox{}