
\noindent
\label{itm:agent}\textbf{Agent}: Autonomous systems that inhabits an environment and act based on predefined rules.

\noindent
\label{itm:abm}\textbf{Agent Based Model (ABM)}: A computer simulation model in which agents interact with each other and their environment to produce emergent behavior. 

\noindent
\label{itm:data-structure}\textbf{Data Structure}: A way of organizing and storing data in a computer so that i can be accessed, manipulated, and modified efficiently. Some common examples are arrays, stacks, and linked lists. 

\noindent
\label{itm:information-visualization}\textbf{Information Visualization}: Field that focuses on creating meaningful and easy to to interpret graphical representations of data.

\noindent
\label{itm:monobehaviour}\textbf{MonoBehaviour}: Base class for Unity scripts. Provides access to event functions such as Start(), Update(), and so on.

\noindent
\label{itm:pooling}\textbf{Pooling}: A technique used in programming to improve performance by reusing objects instead of creating new ones. 

\noindent
\label{itm:prefab}\textbf{Prefab}: A reusable object in Unity that stores a configuration and can be used as a template for creating assets.

\noindent
\label{itm:scrum}\textbf{Scrum}: The scrum agile project management framework provides structure and management of work and is popular among software development teams.

\noindent 
\label{itm:scrum-boards}\textbf{Scrum-boards}: A bulletin board that keeps track of a backlog, the current sprint, and completed stories.

\noindent
\label{itm:story}\textbf{Story}: In the scrum framework, a story is essentially a set of tasks that will result in a new or updated desired functionality/product.  

\noindent
\label{itm:unity}\textbf{Unity}: Cross-platform game engine.

\noindent
\label{itm:unity-asset}\textbf{Unity Asset}: A file  containing reusable content that can be imported into Unity projects. Can be accessed through Unity's official platform or imported via third-party repositories. 

\noindent
\label{itm:user-testing}\textbf{User Testing}: A method of testing and evaluating a product by observing and gathering data from real users. 

\noindent
\label{itm:ui}\textbf{UI}: User Interface (UI) is the point between human-computer interactions. It is what is used for user interactions with the program. 

\noindent
\label{itm:ux}\textbf{UX}: User Experience (UX) refers to the overall experience of the actual user of a product. The goal of good UX design is to create intuitive and enjoyable products.

\noindent
\label{itm:csharp}\textbf{C\#}: C\# is a programming language developed by Microsoft that runs on the platform .NET Framework. C\# is pronounced as "C sharp" and belongs to the programming language family of C.

\noindent
\label{itm:c++}\textbf{C++}: C++ is one of the most popular general purpose programming languages. C++ is pronounced as "C plus plus" and belongs to the programming language family of C.

\noindent
\label{itm:a*}\textbf{A*}: A* is a popular graph traversal and path searching algorithm due to it's completeness and optimal efficiency. A* is used to find the shortest possible path from one specified node to another.

\noindent
\label{itm:repository}\textbf{Repository}: A repository acts as a container that stores a projects files and their individual revision history. 

\noindent
\label{itm:cubic-bézier-curve}\textbf{Cubic Bézier Curve}: 

\noindent
\label{itm:wire-frame}\textbf{Wire Frames}: Wire Frames depict how the UI layout will appear during different stages of the program. 





