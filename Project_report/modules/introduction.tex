% UPDATED BY MARCUS SCHAGERBERG, 2023
% CREATED BY DAVID FRISK, 2016

\section{Section levels}
The following table presents an overview of the section levels that are used in this document. The number of levels that are numbered and included in the table of contents is set in the settings file \texttt{settings.tex}. The levels are shown in Section \ref{Section_ref}.

\begin{table}[H]
\centering
\begin{tabular}{ll} \hline\hline
Name & Command\\ \hline
Chapter & \textbackslash\texttt{chapter\{\emph{Chapter name}\}}\\
Section & \textbackslash\texttt{section\{\emph{Section name}\}}\\
Subsection & \textbackslash\texttt{subsection\{\emph{Subsection name}\}}\\
%Paragraph & \textbackslash\texttt{paragraph\{\emph{Paragraph name}\}}\\
%Subparagraph & \textbackslash\texttt{paragraph\{\emph{Subparagraph name}\}}\\ \hline\hline
\end{tabular}
\end{table}


\section{Purpose}
The purpose of the project is to design and construct a 3D traffic simulation tool with high accessibility that should provide detailed and accessible data that allows the user to evaluate the performance of different road networks and traffic scenarios, and make informed decisions about urban planning and infrastructure. Data should be presented both in real-time, and as post-simulation data through presentation of relevant statistics. By adjusting the parameters of the simulation, the user should be able to witness the effect of their tweaking, and easily see if their changes have a positive or negative impact across relevant environmental dimensions such as traffic flow, travel time, and emissions. 


\section{Related Work} \label{Section_ref}
\subsection{Microscopic Traffic Simulations}
There exists a plethora of different available tools for traffic simulation, which are in turn built upon different underlying models. In Nguyen’s widely cited paper, he classifies the currently available simulations according to the following four categories with regards to their granularity of model: Macroscopic, Microscopic, Mesoscopic, and Nanoscopic. These tools allow researchers to answer complex questions and evaluate different scenarios in both real-time observations and through post-simulation data analysis. \\

Agent-based traffic models position themselves within the Microscopic category and allows for a highly realistic representation of traffic flow, where emergent behaviors such as congestion and bottleneck formation can occur due to the natural occurring interplay of the autonomous agents within the simulation. 

Simulation of Urban MObility (SUMO) is a highly popular and freely available microscopic traffic simulation that was initially developed at the German Aerospace Center (DLR). It provides the users with the ability to model a range of transportation agents, including cars, buses, bicycles, and pedestrians, in both urban environments. The simulation is deterministic by default, but users have the option to introduce stochastic processes in different ways, making it a highly versatile tool for traffic simulation and analysis.\\

The software offers various tools creating networks and editing these through a map editor which can also import and export network data from external sources. In addition to this, SUMO provides the user with features for visualizing the obtained data and analyzing it through various reports and plots. Users can also customize SUMO to accommodate their specific need through the application programming interface (API) and integrate the simulation with other software. SUMO is also capable of modeling emission based on vehicle type and speed. \\

Another popular for simulating traffic is the commercial software PTV Vissim designed by the German-based company PTV group which specializes in mobility and transportation solutions. It is designed to be quick and simple to set up with no scripting required by the user and comes with a highly customizable editor. The software is part of a larger suite named PTV Traffic Suite, which allows it to exchange data and collaborate across multiple platforms. \\

PTV Vissim offers a similar feature list to SUMO but differs in some important areas. Firstly, they are built upon different Car following models. SUMO implements the Krauss model which is based on the idea that drivers adjust their vehicle’s speed and headway based on their perceived safety and comfort. Though a relatively easy to understand model, it has the disadvantage of assuming that the drivers only react to the speed and distance of the vehicle in front of them, and excludes a lot of factors such as the traffic signals, shape of the road, and driver psychology. \\

Meanwhile, PTV Vissim implements the Wiedemann models which share a lot of the same model parameters as the previously mentioned Krauss model but differ in their mathematical formulations and the way they calculate the acceleration of a vehicle. The model also introduces additional parameters, for example, a parameter for setting driver aggressiveness which regulates how risk-taking a driver is willing to be, and a parameter to regulate reaction time. Due to the additional parameters introduced here, the Wiedemann model is considered more realistic compared to the Krauss model, but is at the same time deemed to be more complex and requires a significant amount of parameter calibration. \\

Another crucial difference between the two simulation tools is that SUMO natively only supports graphical representation of a traffic environment in low detailed 2D, while PTV Vissim offers a feature rich 3D visualization. The latter provides a range of tools for customizing the 3D visualization, including options for importing third party 3D models, setting and creating custom textures, and defining various customized visual effects. 
