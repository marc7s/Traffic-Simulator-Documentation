% UPDATED BY MARCUS SCHAGERBERG, 2023
% CREATED BY WOLFGANG AHRENDT, 2021
% WRITTEN BY JAKOB WINDT, 2023

This section will give an account for different results gathered from both our experience with the product, performance benchmarks, and feedback that were collected during user testing. 

\section{Final product}
    The end product is a simulation tool that is capable of simulating vehicles driving in a road network. The road network can either be generated through an imported OSM file, or custom built with the use of the built in Unity Editor. The road network has support for both three and four way intersections. These intersections can either be controlled by traffic lights or stop signs depending on the user's needs. The vehicles themselves are all capable of individually navigating a correctly built road network. The vehicles interacts with their surroundings using the RoadNodes and LaneNodes built into the roads. From these nodes, the vehicle can collect data about whether a node ahead of them is occupied as well as the current speed limit of the road, and if the vehicle needs to brake.

    To use the simulation tool, the users can interact with the simulation through the provided UI. It consists of a main menu, with an included settings page. In addition, there is an overlay menu, that is visible while the simulation is running. With this overlay menu the user can access statistics about an individual vehicle or the entire road network. The statistics are represented through a series of data points and graphs to easily display different statistical aspects of the simulated road network. The user can also change the number of simulated vehicles.

    By using the Unity Editor, the user can as mentioned customize the simulation world. The user can also create roads, add intersections, and add POIs. Furthermore, it is possible to add buses that travel along the bus stops. Parking lots can also be added. These parking lots can either be attached to a road as a street side parking, or as a separate parking lot.

\section{Performance} \label{performance-benchmark}
    % Mention that it was found through the profiler that one of the heaviest operations is updating the vehicle physics
    To test the performance of the tool, an automated testing utility described in section \ref{performance-method} was used. The tests were run on a simulation of the road system for Masthugget, a district of Gothenburg. The tests were run both in the quality mode as well as the performance mode. The test results are compiled in Table \ref{Tab:performance-benchmark-results}.

\begin{table}[ht]
    \caption{Performance test results}
    \centering
    \begin{tabular}{|c|c|c|}
        %% ----- >>> FIRST ROW -----
        \hline
        \multicolumn{1}{|c|}{Configuration} 
        &
        \multicolumn{2}{|c|}{Result}
        \\\hline
        %% ----- <<< FIRST ROW -----
        %% ----- >>> HEADERS -----
        Number of vehicles & Quality FPS & Performance FPS
        \\\hline
        %% ----- <<< HEADERS -----
        %% ----- >>> VALUES -----
        
        5 & 192 & 205
        \\\hline
        50 & 191 & 203
        \\\hline
        100 & 189 & 203
        \\\hline
        250 & 71 & 102
        \\\hline
        500 & 30 & 48
        \\\hline
        1000 & 10 & 25
        
        %% ----- <<< VALUES -----
        \\\hline
    \end{tabular}
    \label{Tab:performance-benchmark-results}
\end{table}

% WRITTEN BY FELIX JÖNSSON, 2023
\section{User tests}
    User testing in software development is an important aspect of a successful end product. It gives the developers insights into how their software is actually perceived in a deployed environment, and how their intended audience interacts with it. This section presents an account of our performed user testing and the feedback received. The testing was conducted with two distinct groups: "experienced testers", and "generic testers". 
    
    % WRITTEN BY MARCUS SCHAGERBERG, 2023
    \subsection{Experienced Tester}
        The first test subject had prior experience in a transportation planning software called PTV Visum, which is the world's most popular software of its kind when it comes to aiding strategic and operative decisions\cite{visum}. Compared to Visum which is set in 2D, the user felt that our tool provided more context, and that it was easier to understand the road network. The test subject pointed out that it was harder to see the road connections and intersections in PTV Visum. However, Visum offered a better view for larger networks. The user appreciated the simulation of individual vehicles compared to Visum, which displays traffic as a value of the number of vehicles per road, and felt that it improved the intuition for smaller networks. Being able to see statistics related to individual vehicles was also pointed out as useful. The test subject felt our tool lacked some customisability that Visum offers, where you can change parameters such as road capacity that affect the simulation. The user mentioned that Visum had a learning curve to understand the buttons and features, which was easier to do in our tool although it does not offer as granular control over the traffic as Visum does. The test subject was not able to use the public transportation feature as it was not done at the time of testing, but expressed interest in it and thought that it was a great idea that Visum lacked. Colour coding the roads based on the congestion level was being implemented at the time of testing, and was something the subject mentioned would also be helpful.

        The tester felt that our tool was easy to understand, and significantly simpler to use for presentation purposes than Visum, which was said to be more technical and harder to understand at first glance. The user had used Visum to demonstrate their solution, and said that it was difficult to find a good way of visualising and presenting it. 
        
        %%It was also thought that our tool would be useful for education purposes, and that it would fit well for teaching different concepts in transportation planning, as it is intuitive and easy to understand what is happening in the simulation.

        Apart from the functionality compared to Visum, the test subject also had feedback regarding the UI. The user thought that it was confusing that there were separate buttons for the statistics, depending on if it was showing individual statistics for a vehicle or aggregated statistics for the entire road network. It was also expressed that it would be easier to navigate using the mouse for adjusting the camera rotation while keeping the keyboard for moving the camera around, and that it felt slow to move around a larger network.
    % WRITTEN BY FELIX JÖNSSON, 2023
    \subsection{Generic Tester}
    The project also conducted user testing on a group we have classified as "generic testers". These individuals possessed no prior experience with any form of simulation software, and the primary objective during these tests was to discern potential deficiencies in the UX design. Some feedback that was acquired included the lack of alternative control schemes. Every tester tried to move around the simulation using the arrow keys, though the tool's movement keys are WASD. Once identified, the camera movement felt good, but they would have liked a clearer communication of the control scheme somehow. This confusion regarding the control scheme also extended to how the user moved between cameras, and some users felt that it was hard to tell the difference between the follow camera and default camera depending on your initial zoom when transitioning to the default camera.

    Furthermore, all generic testers reported a preference for more prominent signaling of the various UI functionalities. This could potentially be achieved through the introduction of informative tooltips, the strategic use of intuitive visual icons, and the implementation of an initial user onboarding walkthrough, that succinctly explains each feature. For example, one tester reported that they felt unsure if they had actually managed to spawn any vehicles when they had pressed the "spawn vehicle"-button, and would have liked some feedback on their input or an explanation of how the spawning functionality actually worked.