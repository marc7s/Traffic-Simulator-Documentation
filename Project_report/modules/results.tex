% UPDATED BY MARCUS SCHAGERBERG, 2023
% CREATED BY WOLFGANG AHRENDT, 2021

% WRITTEN BY JAKOB WINDT, 2023
\section{Final product}
    The end product is a simulation tool that is capable of simulating vehicles driving in a road network. The road network can either be generated through an imported OSM file, or custom built with the use of an editor tool that was custom made during the project. The road network has support for both three and four way intersections. These intersections can either be controlled by traffic lights or stop signs depending on the users need. The vehicles themselves are all capable of individually navigating a correctly built road network. The vehicles interacts with their surroundings using the RoadNode and LaneNodes built into the roads. From these nodes, the vehicle can collect data about whether a node ahead of them is occupied, the current speed limit of the road, or if the vehicles needs to stop due to a red light or stop sign. 

    To use the simulation tool, an UI was created to allow the user to interact with it. It consists of a main menu with an included settings page. In addition, there is an overlay menu that is visible while the simulation is running. With this overlay menu the user can access statistics about an individual vehicle or the entire road network. The statistics are represented through a series of data points and graphs to easily display different statistical aspects of the simulated road network. The user can also spawn and change the amount of vehicles in the network. 

    By using the Unity Editor, the user can as mentioned customize the simulation world. The user can also create roads, add intersections, and add POIs. Furthermore, its possible to add busses that travel in-between the bus stop POIs. Another POI that is possible to add would be parking spots. These parking spots can either be attached to a road as street side parking, or as individual parking lots. 


\section{Performance}
    % Mention that it was found through the profiler that one of the heaviest operations is updating the vehicle physics

\section{User tests}
    % WRITTEN BY MARCUS SCHAGERBERG, 2023
    \subsection{Test subject 1}
        The first test subject had prior experience in a transportation planning software called PTV Visum, which is the worlds most popular software of its kind when it comes to aiding strategic and operative decisions \cite{visum}. Compared to Visum which is set in 2D, the user felt that our tool provided more context and that it was easier to understand the road network. The test subject pointed out that it was easier to see the road connections and intersections in our tool. However, Visum offered a better view for larger networks. The user appreciated the simulation of individual vehicles compared to Visum which displays traffic as a number of vehicles per road, and felt that it improved the intuition for smaller networks. Being able to see statistics related to individual vehicles was also pointed out as useful. The test subject felt our tool lacked some customisability that Visum offers, where you can change parameters such as road capacity that affect the simulation. The user mentioned that Visum had a learning curve to understand the buttons and features, which was easier to do in our tool although it does not offer as granular control over the traffic as Visum does. The test subject was not able to use the public transportation feature as it was not done at the time of testing, but expressed interest in it and thought that it was a great idea that Visum lacked. Colour coding the roads based on the congestion level was being implemented at the time of testing, and was something the subject mentioned would be helpful.

        The test subject felt that our tool was easy to understand, and significantly simpler to use for presentation purposes than Visum, which was said to be more technical and harder to understand at first glance than our tool. The user had used Visum to demonstrate their solution, and said that it was difficult to find a good way of visualising and presenting it. It was also thought that our tool would be useful for education purposes, and that it would fit well for teaching different concepts in transportation planning as it is intuitive and easy to understand what is going on.

        Apart from the functionality compared to Visum, the test subject also had feedback regarding the UI. The user thought that it was confusing that there were separate buttons for the statistics depending on if it was showing individual statistics for a vehicle or aggregated statistics for the entire road network. It was also expressed that it would be easier to navigate using the mouse for adjusting the camera rotation and pitch while keeping the keyboard for moving the camera around, and that it felt slow to move around a larger network.