% Felix 
\section{Agent Based Modeling}
    Agent-Based Modeling (ABM) is a computational modeling approach that facilitates the analysis and simulation of complex systems by depicting a system's individual elements (agents) and their interactions\cite{railsback2019agent}. This method enables researchers to investigate how the combined behavior of a system emerges from the attributes and actions of its individual components. In contrast to conventional models, which typically depend on mathematical tractability and differential equations for portraying behavior from a macroscopic viewpoint, ABMs face fewer restrictions and can encompass more aspects of real-world systems\cite{bonabeau2002agent}. As a result, these models can simulate intricate scenarios without relying on equally complex mathematics. It achieves this with satisfactory, and sometimes, even more precise outcomes compared to models that overlook the individual behaviors ABMs are capable of representing. It should be noted, though, that ABMs can also integrate more sophisticated mathematics and techniques, like neural networks or advanced learning approaches, to more accurately depict the complexities and dynamics of individual agents within the system.

\subsection{Key features}
    ABMs consist of individual agents that interact with each other and their environment. Agents can represent various entities such as organisms, humans, businesses, and so on. These agents are characterized by their uniqueness, local interactions, and autonomy. They can have different attributes such as size, location, and resource reserves, and they interact with their neighbors in a specific "space", such as a geographic area or a network\cite{railsback2019agent}. The mentioned space is typically relatively small in the scope of the total simulation space. Agents act independently and pursue their own objectives, adapting their behavior according to their current state, the state of other agents, and their environment.

\subsection{Emergence and across-level modeling}
    ABMs are particularly useful for studying emergent system behaviors that arise from the interactions and responses of individual components to each other and their environment. This allows researchers to explore how a system's dynamics are linked to the characteristics and behaviors of its individual components. Due to this, ABMs are considered across-level models because they focus on the interactions between the system level and the individual agent level\cite{railsback2019agent}. In these across-level models, the agents' behaviors and decision-making processes are modeled explicitly by the researchers, while the emergent properties of the system as a whole stem from these micro-level interactions that occur at run-time.

    Across-level models allow for a more nuanced understanding of complex systems, as they enable researchers to bridge the gap between micro-level interactions and macro-level outcomes. By capturing the heterogeneity of agents and their responses to their environment, across-level models can shed light on the mechanisms that drive system-level behavior, facilitating the identification of key feedback loops and dependencies within the system.

    Additionally, the models enable researchers to investigate the impact of various factors at both the individual and system levels, such as how changes in individual behaviors or environmental conditions may affect the overall system dynamics. This approach allows for a more thorough exploration of the robustness and adaptability of the system, providing valuable insights for policy development and system management.

\subsection{Advantages and applications}
    ABMs can address complex, multilevel problems that are too difficult to tackle with traditional models. Predator-prey dynamics serve as a classic example of a system traditionally modeled using differential equations and advanced calculus. However, these systems can also benefit from being modeled by an ABM\cite{railsback2020pred-prey}. By employing ABMs to study predator-prey interactions, researchers can gain deeper insights into the adaptive behaviors and decision-making processes of individual organisms. ABMs allow for the representation of heterogeneous agents and the examination of emergent properties arising from their interactions, which can be particularly valuable in understanding the complexities of real-world systems. Applying ABMs can bridge the gap between theoretical and empirical research, highlighting gaps in our knowledge of individual behaviors, and contribute to refining existing theories.

    Although the method appears straightforward to apply, researchers argue that this can create a false impression that the underlying concepts are just as simple to grasp. While ABM might seem technically uncomplicated, it possesses considerable conceptual depth, which frequently results in incorrect utilization.