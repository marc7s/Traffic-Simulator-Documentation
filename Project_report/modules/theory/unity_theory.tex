% WRITTEN BY FELIX JÖNSSON, 2023
\section{Unity}

A game engine is a software framework designed to facilitate the creation of video games by providing the most commonly needed functionalities. These include complex tasks such as physics calculations, animation, rendering, and artificial intelligence. The advantage of using a pre-existing game engine is that it allows developers to focus on the unique aspects of their game and accelerates the development pipeline, as they do not have to code these complex systems from scratch.

Unity is a game engine initially released for Mac OS X in 2005 at the Apple Worldwide Developers Conference\cite{macworld2005}. The CEO of the company has said that the mission was to democratize game development, making it widely accessible to a broad audience\cite{polygon2014}. By 2022, the company had secured a significant market share of 38\%, signifying its wide acceptance and popularity within the industry\cite{slashdata2022}.

Unity can be used to produce both 2D and 3D environments, and it offers native support for a wide variety of platforms and operating systems. Although there is a wealth of underlying theory supporting the engine, of which the most prominent is considered to be the Component-Based Object Model. Development in Unity is centered around so-called GameObjects, which serve as the base class for all entities in Unity scenes. These GameObjects can then receive different components, which can take the form of a wide range of things, such as scripts, textures, cameras, and so on. This pattern allows for a flexible and modular development approach, as many GameObjects can reuse the same component while customizing each component parameter to their specific requirements.

Another central underlying theory is that of Event-Driven Programming and the implementation of a so-called game loop\cite{gameprogrammingpatterns2023}. By using an event-centric communication method for many of its native systems such as input handling and collision detection, components such as scripts can define methods that hook onto the event architecture. These can then respond to specific events such as "OnCollisionEnter", which is called when a collision occurs\cite{unityManualExecutionOrder2023}\cite{unityManualEventFunctions2023}. This Event-Driven approach works in symbiosis with the theory of a game loop. A game loop is a ubiquitous architecture technique used within the game engine sphere. The basic concept is that an update event occurs each frame, also called a tick, where all GameObjects and their components have the opportunity to react and update themselves according to the current state of their environment. Since these updates occur with high frequency, often many times per second, this allows the game to simulate real-time behavior.

